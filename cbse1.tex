\documentclass[12pt,-letter paper]{article}
\usepackage{siunitx}
\usepackage{setspace}
\usepackage{gensymb}
\usepackage{xcolor}
\usepackage{caption}
%\usepackage{subcaption}
\doublespacing
\singlespacing
\usepackage[none]{hyphenat}
\usepackage{amssymb}
\usepackage{relsize}
\usepackage[cmex10]{amsmath}
\usepackage{mathtools}
\usepackage{amsmath}
\usepackage{commath}
\usepackage{amsthm}
\interdisplaylinepenalty=2500
%\savesymbol{iint}
\usepackage{txfonts}
%\restoresymbol{TXF}{iint}
\usepackage{wasysym}
\usepackage{amsthm}
\usepackage{mathrsfs}
\usepackage{txfonts}
\let\vec\mathbf{}
\usepackage{stfloats}
\usepackage{float}
\usepackage{hyperref}
\usepackage{cite}
\usepackage{cases}
\usepackage{subfig}
%\usepackage{xtab}
\usepackage{longtable}
\usepackage{multirow}
%\usepackage{algorithm}
\usepackage{amssymb}
%\usepackage{algpseudocode}
\usepackage{enumitem}
\usepackage{mathtools}
%\usepackage{eenrc}
%\usepackage[framemethod=tikz]{mdframed}
\usepackage{listings}
%\usepackage{listings}
\usepackage[latin1]{inputenc}
%%\usepackage{color}
%%\usepackage{lscape}
\usepackage{textcomp}
\usepackage{titling}
\usepackage{hyperref}
%\usepackage{fulbigskip}   
\usepackage{tikz}
\usepackage{graphicx}
%\usepackage[left=1in, right=2in, top=1in, bottom=1in]{geometry}
\let\vec\mathbf{}
\usepackage{enumitem}
\usepackage{graphicx}
\usepackage{siunitx}
\let\vec\mathbf{}
\usepackage{enumitem}
\usepackage{graphicx}
\usepackage{enumitem}
\usepackage{tfrupee}
\usepackage{amsmath}
\usepackage{amssymb}
\usepackage{mwe} % for blindtext and example-image-a in example
\usepackage{wrapfig}
\graphicspath{{figs/}}
\providecommand{\mydet}[1]{\ensuremath{\begin{vmatrix}#1\end{vmatrix}}}
\providecommand{\myvec}[1]{\ensuremath{\begin{bmatrix}#1\end{bmatrix}}}
\providecommand{\cbrak}[1]{\ensuremath{\left\{#1\right\}}}
\providecommand{\brak}[1]{\ensuremath{\left(#1\right)}}
\providecommand{\norm}[1]{\left\lVert#1\right\rVert}
\providecommand{\abs}[1]{\left\vert#1\right\vert}
\title{Assignment}
\date{\today}
\begin{document}
\maketitle{\centering{\textbf{CBSE MATHEMATICS 2018}}
\begin{enumerate}
\section{\centering Vectors}
\item Find the acute angle between the planes $\overrightarrow{r}.\brak{\hat{i}-2\hat{j}-2\hat{k}}=1$ and $\overrightarrow{r}.\brak{3\hat{i}-6\hat{j}+2\hat{k}}=0$.
\item Find the length of the intercept, cut off by the plane $2{x}+{y}-{z}=5$ on the ${x}$-axis.
\item ${X}$ and ${Y}$ are two points with position vectors $3\overrightarrow{a}+\overrightarrow{b}$ and $\overrightarrow{a}-3\overrightarrow{b}$ respectively.Write the position vector of a point ${Z}$ which divides the lines segment ${XY}$ in the ratio $2:1$ externally.
\item Let $\overrightarrow{a}=\hat{i}+2\hat{j}-3\hat{k}$ and $\overrightarrow{b}=3\hat{i}-\hat{j}+2\hat{k}$be two vectors. Show that the vectors $\brak{\overrightarrow{a}+\overrightarrow{b}}$ and $\brak{\overrightarrow{a}-\overrightarrow{b}}$ are perpendicular to each other.
\item Find the vector equation of the plane which contains the line of intersection of the planes $\overrightarrow{r}.\brak{\hat{i}+2\hat{j}+3\hat{k}}-4=0$,$\overrightarrow{r}.\brak{5\hat{i}+3\hat{j}-6\hat{k}}+8=0$.
\item Find the value of ${x}$ such that the four points with position vectors, ${A}$\brak{3\hat{i}+2\hat{j}+\hat{k}}, ${B}$\brak{4\hat{i}+x\hat{j}+5\hat{k}},
${C}$\brak{4\hat{i}+2\hat{j}-2\hat{k}} and ${D}$\brak{6\hat{i}+5\hat{j}-\hat{k}} are coplanar.
\item Find the vector equation of a line passing through the point $\brak{2,3,2}$ and parallel to the line $\overrightarrow{r} =\brak{-2\hat{i}+3\hat{j}}+\lambda{\brak{2\hat{i}-3\hat{j}+6\hat{k}}}$.Also, find the distance between these two lines.
\item Find the coordinates of the foot of perpendicular ${Q}$ drawn from ${P}$\brak{3,2,1} to the plane $2{x}-{y}+{z}+1=0$.Also, find the distance ${PQ}$ and the image of the point ${P}$ treating this plane as a mirror.
\section{\centering Differentiation}
\item If ${y}=log{\brak{\cos e^x}}$, then find $\dfrac{dy}{dx}$.
\item From the differential equation representing the family of curves ${y} = A\sin x$, by eliminating the arbitrary constant ${A}$.
\item Solve the following differential equation: \begin{align*} \dfrac{dy}{dx}+{y}=\cos {x}-\sin {x} \end{align*}
\item If ${x}=\sin {t}$, ${y}=\sin {pt}$, prove that $(1-x^2)\dfrac{d^2 y}{d x^2}-x\dfrac{dy}{dx}+{p}^{2}{y}=0$.
\item Differentiate $\tan^{-1}\myvec{\frac{\sqrt{1+x^2}-\sqrt{1-x^2}}{\sqrt{1+x^2}+\sqrt{1-x^2}}} $with respect to $\cos^{-1} {x}^{2}$.
\item Solve the differential equation $\dfrac{dy}{dx} = 1+{x}^{2}+ {y}^{2}+{x}^{2}{y}^{2}$, given that ${y}=1$ when ${x}=0$.
\item Find the particular solution of the differential equation $\dfrac{dy}{dx}=\dfrac{xy}{x^{2}+y^{2}}$,given that ${y}=1$ when ${x}=0$.
\item If ${y}=\cbrak{\log{x}}^{x} + {x}^{\log{x}}$, find $\dfrac{dy}{dx}$.
\section{ \centering Matrices}
\item ${A}$ is a square matrix with $\mydet{A}=4$. Then find the value of $\mydet{ A.\brak {adj A}}$.
\item For the matrix ${A}=\myvec{2 & 3\\5 & 7}$, find $\brak{A+A'}$ and verify that it is a symmetric matrix.
\item Using properties of determinants, find the value of ${x}$ for which\begin{align*}\begin{vmatrix}4-x & 4+x & 4+x\\4+x & 4-x & 4+x\\4+x & 4+x & 4-x\end{vmatrix}=0.\end{align*}
\item Using elementary row transformations, find the inverse of the matrix $\myvec{2 & -3 & 5 \\3 & 2 & -4 \\1 & 1 & -2}$.
\item Using matrices, solves the following system of linear equations : \begin{align*}{x+2y-3z}=-4\\{2x+3y+2z}=4\\{3x-3y-4z}=11\\\end{align*}
\section{\centering Integration}
\item Find : \begin{align*}\int{x.\tan^{-1} {x}}{dx}\end{align*}
\item Find : \begin{align*}\int{\dfrac{dx}{\sqrt{5-4x-2x^{2}}}}\end{align*}
\item Find : \begin{align*}\int_{-\pi/4}^{0} \dfrac{1+\tan {x}}{1-\tan {x}}\,{dx}\end{align*}		
\item Prove that $\int_{0}^{a} f{\brak{x}} {dx} = \int_{0}^{a} f{\brak{a-x}} {dx}$, and hence evaluate $\int_{0}^{1} {x}^{2}{\brak{1-x}}^{n} {dx}$.
\item Integrate the function $\dfrac{\cos{\brak{x+a}}}{\sin{\brak{x+b}}}$ w.r.t. ${x}$.
\section{\centering Functions}
\item Let $*$ be an operation defined as $*:{R}\times {R} \rightarrow {R} $ such that ${a}*{b}=2{a}+{b}, {a}, {b} \in {R}$. Check if $*$ is a binary operation. If yes, find if it is associative too.
\item Let ${A}={R}-\cbrak{2}$ and ${B}={R}-\cbrak{1}$. If $f:{A} \rightarrow {B}$ is a function defined by $f({x})=\dfrac{x-1}{x-2}$, show that ${f}$ is one-one and onto. Hence, find ${f}^{-1}$.
\item Show that the relation ${S}$ in the  set  ${A}=\cbrak {{x} \in {Z} : 0 \leq {x} \leq 12}$ given by ${S}= \cbrak{\brak{a,b} :{a}, {b} \in {Z}, \mydet{a-b}\hspace {4pt} \text{is divisible by} \hspace {4pt} 3} $  is an equivalence relation.
\section{ \centering Probability}
\item Out of $8$ outstanding students of a school, in which there are $3$ boys and $5$ girls, a team of $4$ students is to selected for a quiz competition. Find the probability that $2$ boys and $2$ girls are selected.
\item In a multiple choice examination with three possible answers for each of the five questions, what is the probability that a candidate would get four or more correct answers just by guessing?
\item The probabilities of solving a specific problem independently by ${A}$ and ${B}$ are $\dfrac{1}{3}$ and $\dfrac{1}{5}$ respectively. If both try to solve the problem independently, find the probability that the problem is solved.
\item An insurance company insured $3000$ cyclists, $6000$ scooter drivers and $9000$ car drivers. The probability of an accident involving a cyclist, a scooter driver and a car driver are $0.3$, $0.05$ and $0.02$ respectively. One of the insured persons meets with an accident. What is the probability that he is a cyclist?
\section{\centering Algebra}
\item A ladder $13$m long is leaning against a vertical wall. The bottom of the ladder is dragged away from the wall along the ground at the rate of $2$ cm/sec. How fast is the height of the wall decreasing when the foot of the ladder is $5$m away from the wall?
\item Prove that : \begin{align*} \cos^{-1}{\brak{\frac{12}{13}}}+\sin^{-1}{\brak{\frac{3}{5}}}=\sin^{-1}{\brak{\frac{56}{65}}}\end{align*}
\section{ \centering Intersection of Conics}
\item Using integration, find the area of the region bounded by the parabola ${y}^{2}=4{x}$ and the circle $4{x}^{2}+4{y}^{2}=9$.
\item Using the method of the integration, find the area of the region bounded by the lines $3{x}-2{y}+1=0$, $2{x}+3{y}-21=0$ and ${x}-5{y}+9=0$.
\item Using integration, find the area of the smaller region bounded by the ellipse $\dfrac{x^{2}}{9}+\dfrac{y^{2}}{4}=1$ and the line $\dfrac{x}{3}+\dfrac{y}{2}=1$.
\section{\centering Optimization}
\item A manufacturer produces nuts and bolts. It takes $1$ hour of work on machine ${A}$ and $3$ hours on machine $B$ to produce a package of nuts. It takes $3$ hours on machine ${A}$ and $1$ hour on machine ${B}$ to produce a package of bolts. He earns a profit of \rupee~$35$ per package of nuts and \rupee~$14$ per package of bolts. How many packages of each should be produced each day so as to maximise his profit, if he operates each machine for atmost $12$ hours a day ? Convert it into a LPP and solve graphically.
\end{enumerate}`
\end{document}
